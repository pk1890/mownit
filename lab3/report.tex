\section{Napisz w Python bądź C/C++ funkcję rozwiązującą przy pomocy metody bisekcji funkcję f. Funkcja przyjmuje następujące argumenty:
krańce przedziału,
błąd bezwzględny obliczeń.
Funkcja ma zwracać wyznaczone miejsce zerowe i liczbę iteracji potrzebną do uzyskania określonej dokładności. Przetestuj działanie metody dla funkcji podanych na początku instrukcji.
Napisz w Python bądź C/C++ funkcję rozwiązującą przy pomocy metody bisekcji funkcję f. Funkcja przyjmuje następujące argumenty:
krańce przedziału,
błąd bezwzględny obliczeń.
}

\begin{minted}{Python}


def bisect(F, min:float, max:float, accuracy:float, itNo:int):
    center = (max+min)/2
    if(abs(F(center)) < accuracy ):
        print("Used iterations:", itNo)
        return center
    else:
        if(F(center)*F(min) < 0):
            return bisect(F, min, center, accuracy, itNo+1)
        else:
            return bisect(F, center, max, accuracy, itNo+1)

Fa = lambda x: math.cos(x) * math.cosh(x) - 1
Fb = lambda x: 1/x - np.tan(x) #[0, 2pi] z badania przebiegu zmienności wiemy że jedno zero jesst w [0.5, 1] a drugie w [3, 4]
Fc = lambda x: 2**(-x) + math.exp(x) + 2 * math.cos(x) - 6

\end{minted}

\section{Napisz w Python bądź C/C++ funkcję rozwiązującą przy pomocy metody Newtona funkcję f. Funkcja ma wykorzystywać dwa kryteria stopu:
maksymalną liczbę iteracji,
moduł różnicy kolejnych przybliżeń mniejszy od danej wartości eps.
Oprócz przybliżonej wartości pierwiastka funkcja ma zwrócić liczbę iteracji potrzebną do uzyskania określonej dokładności eps. Porównaj zbieżność metody ze zbieżnością uzyskaną dla metody bisekcji.
Napisz w Python bądź C/C++ funkcję rozwiązującą przy pomocy metody bisekcji funkcję f. Funkcja przyjmuje następujące argumenty:
krańce przedziału,
błąd bezwzględny obliczeń.
}

\begin{minted}{Python}



def newton_solve(F, min:float, max:float, accuracy:float, itLim: int):
    x_last = np.infty
    x = min
    i = 0
    while(abs(x - x_last) > accuracy and i < itLim):
        x_last = x
        x = x - (F(x) / derivative(F, x, 1e-10))
        i+=1
    return x

\end{minted}

\section{Napisz w Python bądź C/C++ funkcję rozwiązującą przy pomocy metody Newtona funkcję f. Funkcja ma wykorzystywać dwa kryteria stopu:
maksymalną liczbę iteracji,
moduł różnicy kolejnych przybliżeń mniejszy od danej wartości eps.
Oprócz przybliżonej wartości pierwiastka funkcja ma zwrócić liczbę iteracji potrzebną do uzyskania określonej dokładności eps. Porównaj zbieżność metody ze zbieżnością uzyskaną dla metody bisekcji.
Napisz w Python bądź C/C++ funkcję rozwiązującą przy pomocy metody bisekcji funkcję f. Funkcja przyjmuje następujące argumenty:
krańce przedziału,
błąd bezwzględny obliczeń.
}

\begin{minted}{Python}


def euler_solve(F, min:float, max:float, accuracy:float, itLim: int):
    x1 = min
    x2 = max
    i = 0
    while(abs(x2-x1) > accuracy and i < itLim):
        x3 = (F(x2)*x1-F(x1)*x2)/(F(x2)-F(x1))
        x1 = x2
        x2 = x3
        i+=1
    return x3



\end{minted}


\section{Podsumowanie}

\subsection{Wykresy funkcji}

\begin{figure}
    \centering
    \includegraphics[width=0.8\linewidth]{Fa}
    \caption{Wykres funkcji a}
\end{figure}

\begin{figure}
    \centering
    \includegraphics[width=0.8\linewidth]{Fb}
    \caption{Wykres funkcji b}
\end{figure}
\begin{figure}
    \centering
    \includegraphics[width=0.8\linewidth]{Fc}
    \caption{Wykres funkcji c}
\end{figure}
\begin{lstlisting}
   FA=======
Used iterations bisect: 43
Bisect:  4.730040744862693
Used iterations Newton: 4
Newton:  4.730040744862704
Used iterations Euler : 7
Euler :  4.730040744862704
FB==(1)=====
Used iterations bisect: 38
Bisect:  0.8603335890193193
Used iterations Newton: 5
Newton:  0.8603335890193797
Used iterations Euler : 5
Euler :  0.8603335890193797
FB==(2)=====
Used iterations bisect: 37
Bisect:  3.425618459481484
Used iterations Newton: 5
Newton:  3.4256184594817283
Used iterations Euler : 7
Euler :  3.4256184594817283
FC========
Used iterations bisect: 40
Bisect:  1.8293836019338414
Used iterations Newton: 9
Newton:  1.829383601933849
Used iterations Euler : 11
Euler :  1.8293836019338487 
\end{lstlisting}