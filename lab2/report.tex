
\section{Napisz (Python) i sprawdź funkcję rozwiązującą układ równań liniowych n×n metodą Gaussa-Jordana. Dla rozmiarów macierzy współczynników większych niż 500×500 porównaj czasy działania zaimplementowanej funkcji z czasami uzyskanymi dla wybranych funkcji bibliotecznych.}

\begin{minted}{Python}

    import numpy as np 

    import time
    
    def solve(matrix :np.ndarray, vector: np.ndarray):
        if(matrix.shape[0] != matrix.shape[1]) :
           print("macierz musi byc kwadratowa", matrix[1][0])
           exit()
    
        if(vector.shape[1] != 1 and vector.shape[0] != matrix.shape[0]) :
           print("macierz musi byc kwadratowa", matrix[1][0])
           exit()
    
        # print(matrix)
        # print(np.linalg.inv(matrix))
        # print("====================")
        size = matrix.shape[0]
        result = np.identity(size)
    
        for column in range(0, size):
            for row in range(column, size):
                if matrix.item((row, column)) != 0:
                    matrix[[column, row]] = matrix[[row, column]]
                    result[[column, row]] = result[[row, column]]                 # zamien zawsze tak, aby bu
            value = matrix.item((column, column))
            for row in range(0, size):
                if(row == column): continue
                item = matrix.item((row, column))
                times = (item/value)
                for cell in range(0, size):
                    curr = matrix.item((row, cell))
                    matrix.itemset((row, cell), curr - (matrix.item(column, cell)*times))
                    
                    curr = result.item((row, cell))
                    result.itemset((row, cell), curr - (result.item(column, cell)*times))
    
        for column in range (0, size):
            value = matrix.item((column, column))
            for row in range(0, size):
                matrix.itemset((column, row), matrix.item((column, row)) / value)
                result.itemset((column, row), result.item((column, row)) / value)
    
        
        # print( matrix)
        # print(result)
    
        
        return np.matmul(result, vector)
            # for i in range(0, size):
            # for row in range(0, size):
    
    #print(np.matrix('1 3 2 ; 1 3 3 ').shape)
    #solve(np.matrix('0 2 0 4 ; 0 0 0 3 ; 3 2 0 2 ; 0 2 3 4'))
    
    dim = 600
    m = 1000 * np.random.rand(dim, dim)
    v = 1000 * np.random.rand(dim, 1)
    print(m)
    
    start = time.time()
    reslib = np.linalg.solve(m, v)
    end = time.time()
    
    libtime = end-start
    
    start = time.time()
    myres = solve(m, v)
    end = time.time()
    
    mytime = end-start
    
    
    print(reslib - myres)
    
    print("lib time:", libtime, "mytime: ", mytime)
    
    

\end{minted}{Python}


Wyniki dla macierzy 600x600:

\begin{lstlisting}

	...
	[ 1.32283628e-11]
	[ 4.39959180e-12]
	[ 7.21332022e-12]
	[-2.13793983e-11]
	[-2.40752140e-11]
	[ 4.48685533e-12]
	[-4.05706024e-12]
	[-2.08548734e-11]]
	lib time: 0.3574869632720947 mytime:  195.09919571876526
\end{lstlisting}


\section{Napisz i sprawdź funkcję dokonującą faktoryzacji A=LU macierzy A. Zastosuj częściowe poszukiwanie elementu wiodącego oraz skalowanie.}

\begin{minted}{Python}

import numpy as np 

def doolitleLU(matrix: np.ndarray):
    if(matrix.shape[0] != matrix.shape[1]) :
        print("macierz musi byc kwadratowa", matrix[1][0])
        exit()

    # print(matrix)
    # print(np.linalg.inv(matrix))
    # print("====================")
    size = matrix.shape[0]
    result = np.copy(matrix)

    L = np.ndarray(matrix.shape)
    U = np.ndarray(matrix.shape)

    for col in range(0, size):
        if(matrix.item((col, col)) == 0):
            for row in range(col, size):
                if(matrix.item((row, col)) != 0):
                    matrix[[col, row]] = matrix[[row, col]]
                    break

    for column in range(0, size):
        for row in range(column, size):
            sum = 0
            for i in range(0, column):
                sum += L.item((column, i)) * U.item((i, row))
            U.itemset((column, row), matrix.item(column, row) - sum)    
        for row in range(column, size):
            if(column == row):
                L.itemset((column, column), 1)
            else:
                sum = 0
                for i in range(0, row):
                    sum+= L.item((row, i)) * U.item((i, column))
                L.itemset((row, column), ((matrix.item(row, column) - sum )/ U.item((column, column))))    
    
    np.set_printoptions(suppress=True)

    print(matrix)
    np.set_printoptions(suppress=True)
    print(L)
    print(U)




np.set_printoptions(suppress=True)

doolitleLU(np.matrix('0 1 2 ; 4 5 6 ; 8 9 10'))  
\end{minted}{Python}
\section{Implementacja algorytmu Floyd-Warshall przy pomocy mnożenia macierzy. Język Python bądź C/C++ i prosty przykład z porównaniem wydajnościowym do wersji z pętlami mnożącymi. http://phdopen.mimuw.edu.pl/lato07/paths1.pdf
}
\begin{minted}{Python}
    
import numpy as np 
import math
import time

def min_plus(matrix: np.ndarray, other: np.ndarray):
    if(matrix.shape[0] != matrix.shape[1] != other.shape[0] != other.shape[1]) :
       print("macierz musi byc kwadratowa", matrix[1][0])
       exit()

    dim = matrix.shape[0]
    result = np.ndarray((dim, dim))

    for i in range (0, dim):
        for j in range (0, dim):
            minVal = matrix.item((i, 0)) + other.item((0, j))
            for k in range (1, dim):
                minVal = min(minVal, matrix.item(i, k) + other.item(k, j))
            result.itemset((i, j), minVal)

    return result
    
def shortest_path(matrix: np.ndarray):
    if(matrix.shape[0] != matrix.shape[1]) :
        print("macierz musi byc kwadratowa", matrix[1][0])
        exit()
    res = matrix
    for i in range(0, math.ceil(math.log2(matrix.shape[0]))):
        res = min_plus(res, res)
    return res

def floyd_warshall(matrix: np.ndarray):
    dim = matrix.shape[0]
    for k in range(0, dim):  
        for i in range(0, dim): 
            for j in range(0, dim): 
                matrix.itemset((i, j), min(matrix.item((i, j)) , matrix.item((i, k))+ matrix.item((k, j)))) 

    return matrix

I = np.infty
m = np.array([[0, I, 1, I],
                [2, 0, 5, 1],
                [I, 2, 0, 1],
                [8, I, 12, 0]])
print(m)
mm = min_plus(m, m)
#print (mm)
print(shortest_path(m))
print(floyd_warshall(m))

test = np.random.rand(500, 500)

for i in range(0, 500):
    test.itemset((i, i), 0)

start = time.time()
floyd_warshall(test)
end = time.time()


 
libtime = end-start
start = time.time()
shortest_path(test)
end = time.time()


shorttime = end-start


print("lib time:", libtime, "mytime: ", shorttime)
\end{minted}{Python}


\section{Zaimplementuj 3 metody mnożenia macierzy rzadkich wedle formatów zaprezentowanych pod wskazanym linkiem: https://docs.nvidia.com/cuda/cusparse/index.html (COO/CSR/CSC/ELL) i porównaj jak ilość zer wpływa na czas mnożenia macierzy i ilość zużytej pamięci. Porównanie dokonaj do standardowej metody mnożenia macierzy o złożoności N^3. Język C/C++. Uszczegółowienie: można wykorzystać https://www.it.uu.se/education/phd_studies/phd_courses/pasc/lecture-1 i można pokazać to na przykładzie mnożenia macierz-wektor.
}

\begin{minted}{C++}
    
#include <iostream>
#include <vector>
#include <bits/stdc++.h> 
#include <ctime>
#include <cstdlib>


using namespace std;

class Matrix{
public:
    vector<vector<float>> data;
    size_t dimX, dimY;

    void setVal(size_t x, size_t y, float val){
        data[x][y] = val;
    }

    float getVal(size_t x, size_t y){
        return data[x][y];
    }

    vector<float> operator *(vector<float> vec){
        if(dimY != vec.size()) return vector<float>();
        
        vector<float> res;
        res.resize(dimY);
        for(int i = 0; i < dimY; i++){
            for(int j = 0; j < dimX; j++){
                res[i] += data[i][j] * vec[j];
            }
        }
        return res;
    }

};


struct Coord{
    size_t x, y;
    float val;
};

Coord getCoord(size_t y, size_t x, float val){
    Coord tmp;
    tmp.x = x;
    tmp.y = y;
    tmp.val = val;
    return tmp;
}

bool operator < (const Coord &c1, const Coord &c2) {
    if(c1.y < c2.y) return true;
    else return (c1.y == c2.y && c1.x < c2.x);
}

bool operator > (const Coord &c1, const Coord &c2) {
    if(c1.y > c2.y) return true;
    else return (c1.y == c2.y && c1.x > c2.x);
}


class COOMatrix{
public:

    size_t dimX, dimY;
    vector<Coord> data;

    COOMatrix (size_t x, size_t y){
        dimX = x;
        dimY = y;
    }

    COOMatrix (){
        dimX = 0;
        dimY = 0;
    }

    bool setVal(size_t x, size_t y, float val){
        if(x > dimX || y > dimY) return false;

        for (auto i = data.begin(); i != data.end(); ++i){
            if( i->x == x && i->y == y){
                if(val == 0){
                    data.erase(i);
                }
                else{
                    i->val = val;
                }
                return true;
            }
        }
        Coord tmp;
        tmp.x = x;
        tmp.y = y;
        tmp.val = val;
        data.push_back(tmp);
        sort(data.begin(), data.end());
    }


    vector<float> operator * (vector<float> &vec){
        
        if(this->dimY != vec.size()) return vector<float>();

        vector<float> result;
        result.resize(vec.size());


        for(Coord c : data){
            result[c.y] += c.val * vec[c.x];
        }

        return result;
    } 

    // void print (){
    //     for(int i = 0; i < dimY; i++){
    //         for(int j = 0; j< dimX; j++){
    //             if(data)
    //         }
    //     }
    // }


};

struct CSRCoord{
    float val;
    size_t col;
};

CSRCoord getCrec(size_t col, float val){
    CSRCoord tmp;
    tmp.col = col;
    tmp.val = val;

    return tmp;
}

class CSRMatrix{
    public:

    size_t dimX, dimY;
    vector<CSRCoord> colTable;
    vector<size_t> offset; 




    vector<float> operator * (vector<float> &vec){
        
        if(this->dimY != vec.size()) return vector<float>();

        vector<float> result;
        result.resize(vec.size());


        for(size_t i = 0; i < vec.size(); i++){
            for(size_t j = offset[i]; j < offset[i+1]; j++){
                result[i] += colTable[j].val * vec[colTable[j].col]; 
            }
            
        }



        return result;
    } 

};

class ELLMatrix{
public:
    int MAX_ELEM_ROW;
    size_t dimX, dimY;

    vector<vector<float>> values;
    vector<vector<int>> columns;

    vector<float> operator * (vector<float> &vec){
        
        if(this->dimY != vec.size()) return vector<float>();

        vector<float> result;
        result.resize(vec.size());


        for(size_t i = 0; i < dimY; i++){
            for(size_t j = 0; j < MAX_ELEM_ROW; j++){
                if(columns[i][j] == -1) continue;
                result[i] += vec[columns[i][j]] * values[i][j]; 
            }
        }

        return result;
    } 
};

void print_vec(vector<float> v){
    for(int i = 0; i < v.size(); i++){
        cout << v[i] << " ";
    }
    cout << endl;
}

int main(){

    vector <float> v {1,2,5};
    Matrix m;
    m.dimX = 3;
    m.dimY = 3;
    m.data.push_back(vector<float>{3,0,2});
    m.data.push_back(vector<float>{0,1,2});
    m.data.push_back(vector<float>{2,0,1});


    print_vec(m * v);

    COOMatrix mo;
    mo.dimX = 3;
    mo.dimY = 3;
    mo.data.push_back(getCoord(0, 0, 3));
    mo.data.push_back(getCoord(0, 2, 2));
    mo.data.push_back(getCoord(1, 1, 1));
    mo.data.push_back(getCoord(1, 2, 2));
    mo.data.push_back(getCoord(2, 0, 2));
    mo.data.push_back(getCoord(2, 2, 1));

    print_vec(mo * v);

    CSRMatrix mc;
    mc.dimX = 3;
    mc.dimY = 3;
    mc.colTable.push_back(getCrec(0,3));
    mc.colTable.push_back(getCrec(2,2));
    mc.colTable.push_back(getCrec(1,1));
    mc.colTable.push_back(getCrec(2,2));
    mc.colTable.push_back(getCrec(0,2));
    mc.colTable.push_back(getCrec(2,1));

    mc.offset.push_back(0);
    mc.offset.push_back(2);
    mc.offset.push_back(4);
    mc.offset.push_back(6);
    print_vec(mc * v);

    ELLMatrix ml;
    ml.dimY = 3;
    ml.dimX = 3;

    ml.MAX_ELEM_ROW= 2;
    ml.columns.push_back(vector<int>{0,2});
    ml.columns.push_back(vector<int>{1,2});
    ml.columns.push_back(vector<int>{0,2});

    ml.values.push_back(vector<float>{3,2});
    ml.values.push_back(vector<float>{1,2});
    ml.values.push_back(vector<float>{2,1});

    print_vec(ml * v);

    srand(time(NULL));

    const size_t DIMX = 6000;
    const size_t DIMY = 6000;
    const int  ZEROPROB = 80;
    
    const float LO = 0;
    const float HI = 1000;

    Matrix testm;
    testm.dimX = DIMX;
    testm.dimY = DIMY;

    testm.data.resize(DIMY);

    float tmpV;

    for(int i = 0; i < DIMX; i++){
        for(int j = 0; j < DIMY; j++){
            if(rand() % 100 < ZEROPROB){
                tmpV = 0;
            }
            else
            {
                tmpV = LO + static_cast <float> (rand()) /( static_cast <float> (RAND_MAX/(HI-LO)));
            }
            testm.data[i].push_back(tmpV);
        }
    }

    vector<float> vec;
    vec.resize(DIMY);
    for(int i = 0; i <  DIMY; i ++){
        vec[i] = LO + static_cast <float> (rand()) /( static_cast <float> (RAND_MAX/(HI-LO)));
    }

    clock_t start, end;

    start = clock();
    auto a = testm * vec;
    end = clock();
    a = a;
    
    cout << "\nNormal version time elapsed: " << (end-start)*1.0/CLOCKS_PER_SEC << " ===========\n";

    COOMatrix testo;
    testo.dimX = DIMX;
    testo.dimY = DIMY;
     for(int i = 0; i < DIMX; i++){
        for(int j = 0; j < DIMY; j++){
            if(rand() % 100 < ZEROPROB) continue;
            testo.data.push_back(getCoord(i, j, LO + static_cast <float> (rand()) /( static_cast <float> (RAND_MAX/(HI-LO)))));
        }
     }
    start = clock();
    auto ao = testo * vec;
    end = clock();
    ao= ao;
    
    cout << "\nNormal version time elapsed: " << (end-start)*1.0/CLOCKS_PER_SEC << " ===========\n";

    CSRMatrix testc;
    testc.dimX = DIMX;
    testc.dimY = DIMY;
    int cntr = 0;
    testc.offset.push_back(cntr);
     for(int i = 0; i < DIMY; i++){
        for(int j = 0; j < DIMX; j++){
            if(rand() % 100 < ZEROPROB) continue;
            testc.colTable.push_back(getCrec(j, LO + static_cast <float> (rand()) /( static_cast <float> (RAND_MAX/(HI-LO)))));
            cntr++;
        }
        testc.offset.push_back(cntr);
     }

     start = clock();
    auto ac = testc * vec;
    end = clock();
    ac= ac;
    
    cout << "\nNormal version time elapsed: " << (end-start)*1.0/CLOCKS_PER_SEC << " ===========\n";

    ELLMatrix teste;
    teste.dimX = DIMX;
    teste.dimY = DIMY;
    vector<int> tmpcol;
    vector<float> tmpVal;
    teste.MAX_ELEM_ROW = static_cast<int>(DIMX*ZEROPROB/100);
     for(int i = 0; i < DIMY; i++){
        for(int j = 0; j < teste.MAX_ELEM_ROW; j++){
            tmpVal.push_back(LO + static_cast <float> (rand()) /( static_cast <float> (RAND_MAX/(HI-LO))));
            tmpcol.push_back(rand() % DIMX);
        }
        teste.columns.push_back(tmpcol);
        teste.values.push_back(tmpVal);
        tmpcol.clear();
        tmpVal.clear();
     }

    start = clock();
    auto ae = teste * vec;
    end = clock();
    ae= ae;
    
    cout << "\nELL    version time elapsed: " << (end-start)*1.0/CLOCKS_PER_SEC << " ===========\n";


\end{minted}{C++}

\begin{lstlisting}
    
Normal version time elapsed: 0.408693 ===========

COO version time elapsed: 0.112109 ===========

CSR version time elapsed: 0.094666 ===========

ELL    version time elapsed: 0.588516 ===========

\end{lstlisting}